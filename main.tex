\documentclass{article}
\usepackage[utf8]{inputenc}
\usepackage[spanish]{babel}
\usepackage{listings}
\usepackage{graphicx}
\graphicspath{ {images/} }
\usepackage{cite}

\begin{document}
	
	\begin{titlepage}
		\begin{center}
			\vspace*{1cm}
			
			\Huge
			\textbf{Taller de Memoria}
			
			\vspace{0.5cm}
			\LARGE
			
			\vspace{1.5cm}
			
			\textbf{Adrián Cuervo Gonzalez}
			\vfill
			\begin{figure}[h]
				\includegraphics[width=6cm]{Images/EscudoUdeA.jpg}
				\centering
				\label{fig:EscudoUdeA}
			\end{figure}
			
			\vspace{0.8cm}
			
			\Large
			Departamento de Ingeniería Electrónica y Telecomunicaciones\\
			Universidad de Antioquia\\
			Medellín\\
			Septiembre de 2020
			
		\end{center}
	\end{titlepage}
	
	\tableofcontents
	\newpage
	\section{Que es la memoria de un computador}\label{intro}
	Se le considera memoria a cualquier tipo de hardware capaz de almacenar información por determinado tiempo
	
	\section{Tipos de memoria de computador} \label{contenido}
	Antes de leer mas a fondo sobre el tema conocía levemente las memorias cache, RAM, ROM, Memoria Virtual y el disco duro
	\subsection{Memoria Cache}
	Esta memoria es la mas veloz y la que menos capacidad tiene, también es la memoria encargada de las funciones y las ordenes del computador y sus componentes, la cual se divide en tres memorias (L1,L2,L3), con diferentes capacidades de velocidad y almacenamiento, L1 es la de mayor velocidad pero de menor capacidad, L2 es intermedio entre ambas y L3 es la de mayor capacidad de almacenamiento pero menor velocidad.
	
	\subsection{Memoria RAM}
	La Memoria RAM es la memoria mas importante ya que en esta se guardan los programas mas importantes después de que se inicia, además esta es la que guarda el Sistema Operativo
	
	\subsection{Memoria ROM}
	Es una memoria de solo lectura la cual almacena los datos,programas y funciones para el buen funcionamiento del computador
	
	\subsection{Disco Duro}
	Esta memoria es la que mayor capacidad tiene frente a las otras relacionadas con el computador, en esta se guardan la mayoría de datos de el computador además también 
	
	\subsection{Memoria Virtual}
	esta ubicada en una parte del disco duro y se usa para liberar espacio de la RAM o para guardar en caso de que la RAM este muy congestionada
	
	\section{Gestión de las memorias del computador}
	
	\section{Velocidad de la memoria}
	
	\bibliographystyle{IEEEtran}
	\bibliography{references}
	
\end{document}