\documentclass{article}
\usepackage[utf8]{inputenc}
\usepackage[spanish]{babel}
\usepackage{listings}
\usepackage{graphicx}
\graphicspath{ {images/} }
\usepackage{cite}

\begin{document}
	
	\begin{titlepage}
		\begin{center}
			\vspace*{1cm}
			
			\Huge
			\textbf{Taller de Memoria}
			
			\vspace{0.5cm}
			\LARGE
			
			\vspace{1.5cm}
			
			\textbf{Adrián Cuervo Gonzalez}
			\vfill
			\begin{figure}[h]
				\includegraphics[width=6cm]{Images/EscudoUdeA.jpg}
				\centering
				\label{fig:EscudoUdeA}
			\end{figure}
			
			\vspace{0.8cm}
			
			\LARGE
			Departamento de Ingeniería Electrónica y Telecomunicaciones\\
			Universidad de Antioquia\\
			Medellín\\
			Septiembre de 2020
			
		\end{center}
	\end{titlepage}
	
	\tableofcontents
	\newpage
	\section{Que es la memoria de un computador}\label{intro}
	Se le considera memoria a cualquier tipo de hardware capaz de almacenar información por determinado tiempo, esta información puede ser el sistema operativo, funciones, programas, archivos, datos, etc. Las memorias puedes almacenar cualquier tipo de información y de cualquier cantidad (siempre que la memoria tenga la capacidad y el espacio).
	
	\section{Tipos de memoria de computador} \label{contenido}
	Existen varios tipos de memorias para el computador y la mayoría tiene su función especifica, aquí mencionaremos las memorias mas importantes de la mas veloz y pequeña hasta la de mayor capacidad.
	\subsection{Memoria Cache}
	Esta memoria es la mas veloz y la que menos capacidad tiene, también es la memoria encargada de las funciones y las ordenes del computador, la cual se divide en tres memorias con diferentes capacidades de velocidad y almacenamiento  (L1,L2,L3).
	\begin{itemize}
		\item{L1 (Nivel1) Es la mas veloz pero de menor capacidad, esta funciona a la misma velocidad del procesador.}
		\item{L2 (Nivel2) este nivel es menos veloz que la L1 y mas que la L3, y en capacidad de almacenamiento es menor que la L3 y mayor que la L1.}
		\item {L3 (Nivel3) este nivel es el de mayor capacidad de los tres pero también es el mas lento pero sigue siendo uno de los mas veloces comparado con las siguientes memorias.}
	\end{itemize}
	
	\subsection{Memoria RAM}
	Es una memoria que se utiliza para leer y escribir datos, la memoria RAM es la memoria mas importante ya que en esta se guarda los procesos temporales e instrucciones para que se ejecuten programas instalados, existen varios tipos de esta memoria como la DRAM, la SDRAM y la RDRAM.\cite{ram}
	\begin{itemize}
		\item{DRAM:}
		este tipo de RAM es de las mas económicas pero tiene muy poca velocidad para el procesamiento de datos.   
		
		\item{SDRAM:}
		este tipo de RAM es mas eficiente ya que opera a la misma velocidad que la placa madre o MotherBoard.
		
		\item{RDRAM:}
		esta se puede decir que es la mas veloz y eficiente pero es demasiado costosa por la dificultad para su fabricación.
	\end{itemize}
	
	\subsection{Memoria ROM}
	Es una memoria de solo lectura la cual almacena los datos, programas y funciones para el arranque del computador y cargar su sistema operativo\cite{rom}
	
	\subsection{Memoria Virtual}
	Se usa para liberar espacio de la RAM o para guardar en caso de que la RAM este llena, esta usa espacio del disco duro para ampliarse.
	
	\subsection{Disco Duro}
	Esta memoria es la que mayor capacidad tiene frente a las otras, en esta se guardan la mayoría de datos del computador como programas, archivos, documentos, etc. Existen tres tipos de disco duro: mecánicos, de estado solido e híbridos (mezcla de los dos anteriores). Siendo el de estado solido el mas veloz por su composición y funcionamiento.
	
	\subsection{Memorias Auxiliares}
	Estas son memorias externas que sirven para guardar archivos y datos que se encuentran en el computador, pueden ser retiradas sin causar fallas. Algunas de ellas son: USB, MicroSD, CD, DVD, etc.
	
	\section{Gestión de las memorias del computador}
	La gestión de las memorias se hace mediante un controlador de memoria que se encarga de comunicarse con el microprocesador, además este también es el que se encarga de buscar el tipo de memoria que necesita y que función va a realizar en ella.
	Este controlador se puede encontrar dentro del microprocesador o en un chip llamado MCH ubicado en la placa madre.\cite{memoria}
	
	\section{Velocidad de la memoria}
	La velocidad de la memoria depende de varios factores como la latencia, la frecuencia y la arquitectura de los componentes.\\
	La latencia son los retardos producidos por el ingreso,
	la lectura o el transporte de datos, la frecuencia esta definida por el reloj del procesador que con cada ciclo se realiza una transferencia de datos y por ultimo esta la arquitectura de los componentes como la memoria RAM,
	El microprocesador y el Bus; estos afectan o mejoran la velocidad respecto a sus características.\cite{velocidad}\\
	
	La velocidad es importante ya que hace que el computador no tarde tanto para leer, procesar y transportar la información y las funciones, ya que cada proceso tiene varios subprocesos lo que hace que tarde cada vez mas con cada uno de ellos
	\vfill
	\newpage
	
	
	
	\bibliographystyle{IEEEtran}
	\bibliography{references}
	
\end{document}