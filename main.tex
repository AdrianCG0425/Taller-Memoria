\documentclass{article}
\usepackage[utf8]{inputenc}
\usepackage[spanish]{babel}
\usepackage{listings}
\usepackage{graphicx}
\graphicspath{ {images/} }
\usepackage{cite}

\begin{document}
	
	\begin{titlepage}
		\begin{center}
			\vspace*{1cm}
			
			\Huge
			\textbf{Taller de Memoria}
			
			\vspace{0.5cm}
			\LARGE
			
			\vspace{1.5cm}
			
			\textbf{Adrian Cuervo Gonzalez}
			
			\vfill
			\begin{figure}[h]
				\includegraphics[width=6cm]{Images/EscudoUdeA.jpg}
				\centering
				\label{fig:EscudoUdeA}
			\end{figure}
			
			\vspace{0.8cm}
			
			\Large
			Despartamento de Ingeniería Electrónica y Telecomunicaciones\\
			Universidad de Antioquia\\
			Medellín\\
			Septiembre de 2020
			
		\end{center}
	\end{titlepage}
	
	\tableofcontents
	\newpage
	\section{Que es la memoria de un computador}\label{intro}
	Se le concidera memoria a cualquier tipo de hardware capas de almacenar informacion por determinado tiempo
	
	\section{Tipos de memoria de computador} \label{contenido}
	Antes de leer mas a fondo sobre el tema conocia levemente las memorias cache, RAM, ROM, Memoria Virtual y el disco duro
	\subsection{Memoria Cache}
	
	
	\subsection{Memoria RAM}
	en esta la memoreia es esto y aquello ....... llenar
	También es posible citar libros \cite{dirac} o documentos en línea \cite{knuthwebsite}.\\\\
	Revisar en la última sección el formato de las referencias en IEEE.
	
	\subsection{Memoria ROM}
	
	
	\subsection{Disco Duro}
	
	\subsection{Memoria Virtual}
	
	Despues de leer sobre los diferentes tipos de memorias conoci las siguientes
	
	\subsection{Memoria DRAM}
	
	\subsection{Memoria SRAM}
	
	\subsection{Memoria SAM}
	
	
	\section{Gestion de las memorias del computador} \label{contenido}
	\section{Velocidad de la memoria}
	
	\bibliographystyle{IEEEtran}
	\bibliography{references}
	
\end{document}